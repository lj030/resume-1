%-------------------------
% Resume in Latex
% Author : Sourabh Bajaj
% License : MIT
%------------------------

\documentclass[letterpaper,11pt]{article}

\usepackage{latexsym}
\usepackage[empty]{fullpage}
\usepackage{titlesec}
\usepackage{marvosym}
\usepackage[usenames,dvipsnames]{color}
\usepackage{verbatim}
\usepackage{enumitem}
\usepackage[hidelinks]{hyperref}
\usepackage{fancyhdr}
\usepackage[english]{babel}
\usepackage{tabularx}
\usepackage{CJKutf8}
\input{glyphtounicode}

\pagestyle{fancy}
\fancyhf{} % clear all header and footer fields
\fancyfoot{}
\renewcommand{\headrulewidth}{0pt}
\renewcommand{\footrulewidth}{0pt}

% Adjust margins
\addtolength{\oddsidemargin}{-0.5in}
\addtolength{\evensidemargin}{-0.5in}
\addtolength{\textwidth}{1in}
\addtolength{\topmargin}{-.5in}
\addtolength{\textheight}{1.0in}

\urlstyle{same}

\raggedbottom
\raggedright
\setlength{\tabcolsep}{0in}

% Sections formatting
\titleformat{\section}{
  \vspace{-4pt}\scshape\raggedright\large
}{}{0em}{}[\color{black}\titlerule \vspace{-5pt}]

% Ensure that generate pdf is machine readable/ATS parsable
\pdfgentounicode=1

%-------------------------
% Custom commands
\newcommand{\resumeItem}[2]{
  \item\small{
    \textbf{#1}{#2 \vspace{-2pt}}
  }
}

% Just in case someone needs a heading that does not need to be in a list
\newcommand{\resumeHeading}[4]{
    \begin{tabular*}{0.99\textwidth}[t]{l@{\extracolsep{\fill}}r}
      \textbf{#1} & #2 \\
      \textit{\small#3} & \textit{\small #4} \\
    \end{tabular*}\vspace{-5pt}
}

\newcommand{\resumeSubheading}[4]{
  \vspace{-1pt}\item
    \begin{tabular*}{0.97\textwidth}[t]{l@{\extracolsep{\fill}}r}
      \textbf{#1} & #2 \\
      \textit{\small#3} & \textit{\small #4} \\
    \end{tabular*}\vspace{-5pt}
}

\newcommand{\resumeSubheadingOneline}[3]{
  \item\begin{tabular*}{0.97\textwidth}[t]{l@{\extracolsep{\fill}}c@{\extracolsep{\fill}}r}
      \textbf{#1} & #2 &
      \textit{\small#3}
    \end{tabular*}\vspace{-5pt}
}

\newcommand{\resumeSubTitle}[1]{
  \vspace{-10pt} \\
  {#1} \\
  \vspace{-5pt}
}

\newcommand{\resumeSubSubheading}[2]{
    \begin{tabular*}{0.97\textwidth}{l@{\extracolsep{\fill}}r}
      \textit{\small#1} & \textit{\small #2} \\
    \end{tabular*}\vspace{-5pt}
}

\newcommand{\resumeSubItem}[2]{\resumeItem{#1}{#2}\vspace{-4pt}}

\renewcommand{\labelitemii}{$\circ$}

\newcommand{\resumeSubHeadingListStart}{\begin{itemize}[leftmargin=*]}
\newcommand{\resumeSubHeadingListEnd}{\end{itemize}}
\newcommand{\resumeItemListStart}{\begin{itemize}}
\newcommand{\resumeItemListEnd}{\end{itemize}\vspace{-5pt}}

%-------------------------------------------
%%%%%%  CV STARTS HERE  %%%%%%%%%%%%%%%%%%%%%%%%%%%%


\begin{document}
\begin{CJK}{UTF8}{gkai}

%----------HEADING-----------------
\begin{tabular*}{\textwidth}{l@{\extracolsep{\fill}}r}
  \large 姓名: 李健& \large 邮箱 : lj\_cs@foxmail.com\\
  \large 出生年月: 1995.09  & \large 电话 : 13120362233\\
\end{tabular*}


%-----------EDUCATION-----------------
\section{教育经历}
  \resumeSubHeadingListStart
    \resumeSubheadingOneline
      {北京航空航天大学}{计算机科学与技术硕士}{2015.09 -- 2018.04}
    \resumeSubheadingOneline
      {长沙理工大学}
      {\quad网络工程学士}{2011.09 -- 2015.06}
  \resumeSubHeadingListEnd


%-----------EXPERIENCE-----------------
\section{工作经历}
  \resumeSubHeadingListStart
    \resumeSubheadingOneline
      {快手}
      {\quad\quad\quad\quad\quad音视频技术部数据研发组研发工程师}{2019.10 -- 至今}
      \resumeItemListStart
        \resumeItem{}
          {实时数据开发和数仓建设}
        \resumeItem{}
          {音视频异常检测和归因系统设计与开发}
        \resumeItem{}
          {指标管理以及相应的查询订阅服务、质量管理能力建设}
        \resumeItem{}
          {数据春节等活动重保}
      \resumeItemListEnd
      \resumeSubheadingOneline
      {百度}
      {\quad\quad\quad\quad\quad\quad\quad知识图谱部离线架构组研发工程师}{2018.04 -- 2019.09}
      \resumeItemListStart
        \resumeItem{}
          {百度知识库数据存储系统Restful批处理接口,流计算场景下的RPC服务开发}
        \resumeItem{}
          {知识计算框架实体检索模块优化以及算子网关开发}
        \resumeItem{}
          {百度知识图谱SPO收录机制的开发}
        \resumeItem{}
          {Hadoop、HBase、ES、MongoDB集群和服务的维护工作}
      \resumeItemListEnd
      
% --------Multiple Positions Heading------------
%    \resumeSubSubheading
%     {Software Engineer I}{Oct 2014 - Sep 2016}
%     \resumeItemListStart
%        \resumeItem{Apache Beam}
%          {Apache Beam is a unified model for defining both batch and streaming data-parallel processing pipelines}
%     \resumeItemListEnd
%    \resumeSubHeadingListEnd
%-------------------------------------------


  \resumeSubHeadingListEnd

%-----------SKILLS-----------------
\section{专业技能}
  \resumeSubHeadingListStart
    \resumeSubItem{}
      {熟悉Java栈、Python工程开发; 熟悉sql, shell, Javascript等脚本开发}
    \resumeSubItem{}
      {熟悉流处理框架Flink, 以及Hadoop生态Yarn、Hive、Spark}
    \resumeSubItem{}
      {熟悉OLAP引擎Clickhouse, Druid, NoSQL存储引擎HBase, Elasticsearch}
    \resumeSubItem{}
      {熟悉分布式数据系统的基本原理,并且有相应的应用调优经验}
    \resumeSubItem{}
      {熟悉微服务架构,具有Spring Boot, gRPC应用开发经验}
    \resumeSubItem{}
      {熟悉常用的数据库、消息队列和缓存技术, 例如MySQL、Redis、Kafka、ZooKeeper等}
    \resumeSubItem{}
      {具有一定的数仓建设和数据治理的经验,包括有模型设计以及流程规范等}
  \resumeSubHeadingListEnd


%-----------PROJECTS-----------------
\section{项目经验}
  \resumeSubHeadingListStart
    \resumeSubItem{快手-音视频体验优化实时数仓}{}
    \resumeItemListStart
        \resumeItem{}
          {项目描述: 面向快手主站和海外产品下音视频全链路上报数据, 基于音视频指标体系构建实时数仓, 支持CDN质量管理、排障、实时A/B、OLAP数据分析、策略算法模型及异常检测归因等数据应用}
        \resumeItem{}
          {主要工作: 1. 快手直播全链路质量实时数仓建设, 从主播推流、源站、CDN分发、拉流 2. 生产侧实时数仓建设 3. 流批一体能力建设 4. 活动时数据重保}
        \resumeItem{}
          {难点: 1. GB/S 千万QPS级别下的实时数据开发、作业性能优化和表优化 2. 音视频业务场景复杂带来的链路长、数据不完全、数据不统一等难点 3. 需要高效的流批一体以及实时数据回放机制 4. 快手活动给数据任务保障带来的挑战}
        \resumeItem{}
          {相关技术: Java、Flink、Clickhouse、Druid、ES、Hadoop}
    \resumeItemListEnd
    \resumeSubItem{快手-音视频异常检测与归因系统}{}
    \resumeItemListStart
    \resumeItem{}
      {项目描述: 人工故障发现、定位、止损恢复耗时久不及时,
      需要智能的异常检测、归因、自愈的自动化能力提高故障发现和止损效率、保障音视频用户体验。系统目前共接入共计160+个指标,覆盖了快手主站、海外Kwai以及其他音视频中台支持产品下点播、生产、直播、网络库、静态资源下载、客户端等主题域下10+业务,千万量级监控曲线; 异常检测准确率达83\%,召回率达82\%,异常发现时间中位数12min。归因准确率80\%, 直接定位根因比例50\%, 监控人效率优化60\%; 除上述业务上收益外,系统以流程编排框架为基础,并通过插件化实现自动监控、算法白盒化等机制,建设了20+算法能力,开发人效节省30PD/月, 通过封装PySpark算子支持分布式训练,保证256个训练任务24h完成训练同时节省计算资源使用1500CU}
    \resumeItem{}
      {主要工作: 系统历经了两个大阶段,第一阶段基于人工阈值类策略,第二阶段升级为无监督智能算法策略,负责从0到1的系统工程构建、产品建设、SDK封装}
    \resumeItem{}
      {难点: 1. 工程和存储架构的设计和演进支持系统的可扩展性、可用性、可维护性,支持通用能力的复用 2. 高效的算法策略开发、上线和排障框架设计 3. 多维空间下海量指标时序的异常检测以及归因分析能力建设}
    \resumeItem{}
      {相关技术: Python、流程编排、Spark、Hadoop、gRPC, Spring Boot}
    \resumeItemListEnd
    \resumeSubItem{快手-音视频指标管理服务}{}
    \resumeItemListStart
    \resumeItem{}
      {项目描述: 音视频实时数据治理下的主要服务,面向音视频业务下存在指标取数效率低、指标口径难以统一、生产消费链路耦合、消费场景管控缺失的痛点,提供统一的指标接入以及查询服务; 通过构建数据生产链路血缘、面向生产环节的数据质检以及面向上层应用的实时数据可用性定义和接口,建立数据质量管理和保障机制; 服务接入10+种业务场景、140+指标、100+维度、22张数据表 支持了异常检测归因归因、大屏等多个业务, 可节省1day+指标接入时长;推动实时任务可用性优化,实时任务可用性分数P95达82.8\%, P0任务可用性分数达到99.67\%)
      }
    \resumeItem{}
      {主要工作: 1. 指标查询和订阅服务开发 2、数据质量SLA体系建设 }
    \resumeItem{}
      {难点: 1、指标、维度统一管理、与数据表解耦 2、统一的查询用户视图 3、数据质量保障}
    \resumeItem{}
      {相关技术: 指标维度建模、gRPC、Spring Boot}
    \resumeItemListEnd
    \resumeSubItem{百度-知识库数据存储系统}{}
    \resumeItemListStart
    \resumeItem{}
      {项目描述: 基于HBase(300个节点)开发的一套PB级别知识库数据存储系统, 支撑百度知识图谱PB级全量数据和每天约200TB的增量更新, 主要实现以下功能, 1)提供大规模结构化数据存储服务 2)封装了常见的HBaseAPI操作: 包括数据批量和单条的CRUD操作,简化了用户使用Hbase的成本 3)在Hbase基础上增加了一套权限控制方案 4)提供了数据备份、容灾机制 5)为数据提供了干预、回灌等功能,为数据生产提供了遍历}
    \resumeItem{}
      {主要工作: 1. RESTful和RPC接口开发 2. 权限系统升级和维护 3. 运维工具开发} 
    \resumeItem{}
      {相关技术: 1. Java多线程, 网络编程 2. HBase客户端开发 3. RPC服务}
    \resumeItemListEnd
    \resumeSubItem{百度-知识计算框架KB检索模块优化}{}
    \resumeItemListStart
    \resumeItem{}
      {项目描述: 知识计算框架对知识的计算, 存储功能进行了封装, 实现了DAG任务流。检索模块为其中的一个工作任务(组件),支持SPO消歧等任务, 在Elasticsearch(48台机器)索 引(10亿文档)中对数据进行查询; 从搜索策略、ES索引调整、集群部署优化三个方面进行了优化,服务qps由100提升到700}
    \resumeItem{}
      {主要工作: KB检索模块优化方案设计和代码开发} 
    \resumeItem{}
      {相关技术:  1. elasticserach, mongodb 2. python开发}
    \resumeItemListEnd
  \resumeSubHeadingListEnd

%
%--------PROGRAMMING SKILLS------------
%\section{Programming Skills}
%  \resumeSubHeadingListStart
%    \item{
%      \textbf{Languages}{: Scala, Python, Javascript, C++, SQL, Java}
%      \hfill
%      \textbf{Technologies}{: AWS, Play, React, Kafka, GCE}
%    }
%  \resumeSubHeadingListEnd


%-------------------------------------------
\end{CJK}
\end{document}
